\section{Method} \label{method}
The filter considers \textit{line scetching} and \textit{shading} separately.
The following subsections shed some light on how both stages work precisely.

The input for the filter is a regular RGB-image $I_{rgb}$. In the first step a
grayscale image $I_g$ is calculated using the $Y$-channel of the
$YUV$-transformed image $I_{iuv}$. $I_g$ now serves as input for the future
steps.

\subsection{Line Scetching}
One prominent task of the filter is to create sketchy looking outlines. The main
objective here is to mimic freehand sketching, which is typically done by
drawing short straight lines aligned to the outlines. The outlines are detected
using a simple gradient operator. Then each pixel is assigned to a line
direction and finally the pixel is drawn as part of its line.

\paragraph{Gradient Image}
The Outlines are detected and stored in the image $G$ using gradient magnitudes:
\begin{align*}
  G = ((\partial_x G)^2 + (\partial_y G)^2)^{\frac{1}{2}}
\end{align*}
TODO Bild verweis!!

To classify each pixel to a line direction it would be possible to just use
gradient directions, however those are typically noisy. Therefore a very stable
convolution based direction classification is used. 

\paragraph{Line Convolution Filter}
Given just the gradient magnitudes $G$ for each line direction $i \in
\lbrace 1,\cdots,N\rbrace$ where $N$ is the total number of lines, the convolution between
$G$ and each line segment $\mathscr{L}_i$ is calculated.
\begin{align*}
  G_i = \mathscr{L}_i * G
\end{align*}

The value $G_i(p)$ of pixel $p$ will be very big, if that pixel lies directly on
a line in $G$ (edge) and if $\mathscr{L}_i$ is following this line, such that only big
values are collected in the convolution. If the pixel doesn't lie on or close to
a line it can not gather high values and therefore stay dark. Pixels which lie
very close to edges can still gather some brightness if the
line segment $\mathscr{L}_i$ intersects the edge. This way lines in $G$ which
follow the direction of $\mathscr{L}_i$ show up and slightly overshoot in $G_i$.

Now to actually draw the lines, each pixel selects its maximum value from all
$G_i$:
\begin{align*}
 L = \max(\lbrace G_i\rbrace) \quad \forall i \in  \lbrace1,\cdots,N\rbrace
\end{align*}

TODO: Bilder einfügen!

\subsection{Shading}
The other important step in creating a believable pencil sketch image from a
natural image is to produce a hatching texture to create the shading. This is
done in two steps. First the histogram of the input image $I_g$ is matched to a
histogram model that was derived in \cite{mainPaper}. This way the tone
distribution is forced to correspond to tone distributions that were measured in
real pencil sketch images. Then the image of a given hatching pattern is used to
render the hatching texture for the input image. 

\paragraph{Histogram Matching}
Tones in natural images do not follow any specific pattern. In pencil drawings
however the tones are basically created only by two basic tones, namely the
white paper and the graphite strokes in different strengths. Heavy strokes are
used in very dark areas, mid tone strokes are used to produce impression of 3D
layered information and in bright areas the paper is just left white.
\autoref{fig:real-histograms} shows the tone distributions of some real pencil
sketches. One can easily see the three regions, the peak in the dark regions,
which represent the heavy strokes, the constant distribution in the mid tones,
which are used for the layering and very much bright pixels, originating from
the white paper, that was just left blank.

\begin{figure}[htb]
  \centering
  \includegraphics[width=0.4\textwidth]{images/real-histograms.png}
  \caption{Examples for real pencil sketches and their measured tone
    distributions. Note: This image was taken from \cite{mainPaper}}
  \label{fig:real-histograms}
\end{figure}

\cite{mainPaper} used this observation to create a parametric histogram model
for pencil drawings which consists of three functions, which represent those
three tone levels:

For the bright part of the histogram they use a Laplacian distribution with a
peak at the brightest value. This adds some variation in the bright areas, which
originate from slight illumination variances or the use of a eraser.

\begin{align}
  p_1(v) = \frac{255}{\sigma_b} e ^{-\frac{255-v}{\sigma_b}} \label{eq:p_1}
\end{align}

The parameter for this function is just $\sigma_b$, which controls the sharpness of
the function. This distribution can be seen on the very right of the histograms
in \autoref{fig:real-histograms}.

The mid layer is composed of strokes with different pressures and therefore in
different gray levels. So the distributions of those gray levels is equally
distributed as indicated in the histograms in \autoref{fig:real-histograms}. To
represent this part a constant function was chosen to use all those possible
gray levels.

\begin{align}
  p_2(v) = \begin{cases} \frac{1}{u_b - u_a} & \text{if } u_d < v \leq u_b\\
    0 & \text{otherwise}
  \end{cases} \label{eq:p_2}
\end{align}

The controlling parameters for this function are the range boundaries $u_d$
and $u_b$.

Finally the dark region which shows up as this bell shaped peak in the dark
regions in \autoref{fig:real-histograms} is represented as a Gauss-curve. The
position and shape of the dark regions depend on the maximum pressure an artist is using,
and the softness of the pencil that is used.

\begin{align}
  p_1(g) = \frac{1}{\sqrt{2\pi \sigma_d}} e^{-\frac{(v-\mu_d)^2}{2\sigma_d^2}} 
  \label{eq:p_3}
\end{align}

The width of the bell is controlled with the parameter $\sigma_d$ and the
position with $\mu_d$.

Plots for the three functions can be seen in \autoref{fig:p}.

\begin{figure}[htb]
  \centering
  \includegraphics[width=0.4\textwidth]{images/p_i.png}
  \caption{Plots of the three functions $p_i$. Note: picture taken from
    \cite{mainPaper}.}
  \label{fig:p}
\end{figure}

The final tone distribution is now simply composited out of those three
function by creating a weighted sum from $p_1$, $p_2$ and $p_3$:
\begin{align}
  p(v) = \frac{1}{Z} \sum_{i=1}^{3}\omega_i p_i(v)
  \label{eq:p}
\end{align}
Where $Z$ is a normalization factor to make $\int_0^{255}p(v)dv = 1$ and the
$\omega_i$ are weighting parameters which can used to weight the importance of
the functions.

In \cite{mainPaper} they learned the parameters for those functions from a set
of different styled pencil sketches using Maximum Likelihood Estimation. We skipped this part and left those
parameters to be controlled by the user.

TODO matching .. matching
\paragraph{Textureing}
Raphael
