\documentclass{utue} %utue.cls required for Uni Tuebingen corporate design
\usepackage{amsmath}
\usepackage[hidelinks]{hyperref}
\def\sectionautorefname{Section}
% Values for title generation
\title{Cuda Implementation of Advanced Pencil Sketch Filter}
\author{Andreas Altergott, Raphael Braun, Stefan Burnicki}
\date{\today}

% Subtitle is optional. It represents what kind of work you did.
\subtitle{Practical course:\\Massive Parallel Programming SS2014}
\begin{document}

% You can place a teaser as follows. (Otherwise, just uncomment the following part)
\teaser{
    \includegraphics[width=.9\textwidth]{images/teaser.jpg}
    \caption{\label{fig:scribble}A application of the presented method is to
  consistently propagate user input scribbles through a video. Optical flow (c) is
calculated from the input video (a) using sparse feature matching. Then the
scribbles (b) are spread in space and time to generate (d)}
}     

% Creates title of document and additional title page.
\maketitle 
 
\section*{Abstract} 
This paper describes the Cuda implementation of an advanced pencil sketch
filter, based on the work of Cewu Lu, Li Xu and Jiaya Jia \cite{mainPaper}.  It
explains the used methods of the original paper and discusses the derived
implementations for parallel computing. Details of the most complex kernels are
explained throughout the paper, as well as some important modifications to improve
performance. The result of this work are extremely fast computation times and
very pleasing looking pencil sketch images.


\section{Introduction}
Our Paper is: \cite{mainPaper}


\section{Previous Work}
This work is based on \cite{mainPaper}, as this paper describes the filter that
is implemented. The filter requires quite some parameters. In \cite{mainPaper}
they use a parameter lerning approach to automatically select those parameters.
In this work the filter is kept slightly simpler than in the origin paper, e.g.
the line filter is slightly simplified and the filter parameters are left as
user controllable options. 

Another GPU-sketch-filter is described in \cite{Lu:2010:IPS}, though it uses the
a regular shaderpipeline. Based on the neighbourhood informations of each pixel
it calculates weather to place a stroke at this point or not. The strokes are
then rendered as stroke-textures. The strokes are made in three different detail
layers based on the gradientmagnitude. Furthermore it is possible to filter
animations using optical flow. The greatest difference to the filter that we
implement is that they calculate each individual pencil or brush stroke based on
the image and then render those strokes using stroke-textures, and our method
uses traditional filters to alter the original image. Their filter is quite
versatile, as it allows different painting stiles by changing the
stroke-textures and weighting the detail layers.


A partially GPU based method is described in \cite{Benard:2012:ASC}. However
they concentrate on rendering temporal coherent line drawing animations from a
given 3D-scene. First they find active contours that are then adapted throughout
the animation. The result pictures are rendered by parameterizing the and then
add a certain style to the path. While this method yields amazing results for
very stylistic animations, it needs 3D input data.


\section{Method} \label{method}
The filter considers \textit{line scetching} and \textit{shading} separately.
The following subsections shed some light on how both stages work precisely.

The input for the filter is a regular RGB-image $I_{rgb}$. In the first step a
grayscale image $I_g$ is calculated using the $Y$-channel of the
$YUV$-transformed image $I_{yuv}$. $I_g$ now serves as input for the future
steps.

\subsection{Line Sketching}
One prominent task of the filter is to create sketchy looking outlines. The main
objective here is to mimic freehand sketching, which is typically done by
drawing short straight lines aligned to the outlines. The outlines are detected
using a simple gradient operator. Then each pixel is assigned to a line
direction and finally the pixel is drawn as part of its line.

\paragraph{Gradient Image}
The Outlines are detected and stored in the image $G$ using gradient magnitudes:
\begin{align*}
  G = ((\partial_x G)^2 + (\partial_y G)^2)^{\frac{1}{2}}
\end{align*}
In the center of \autoref{fig:sketch-steps} you can see how $G$ looks like.

\paragraph{Line Convolution Filter}
Given just the gradient magnitudes $G$ for each line direction $i \in
\lbrace 1,\cdots,N\rbrace$ where $N$ is the total number of lines, the convolution between
$G$ and each line segment $\mathscr{L}_i$ is calculated.
\begin{align*}
  G_i = \mathscr{L}_i * G
\end{align*}

The value $G_i(p)$ of pixel $p$ will be very big, if that pixel lies directly on
a line in $G$ (edge) and if $\mathscr{L}_i$ is following this line, such that only big
values are collected in the convolution. If the pixel doesn't lie on or close to
a line it can not gather high values and therefore stay dark. Pixels which lie
very close to edges can still gather some brightness if the
line segment $\mathscr{L}_i$ intersects the edge. This way lines in $G$ which
follow the direction of $\mathscr{L}_i$ show up and slightly overshoot in $G_i$.

Now to actually draw the lines, each pixel selects its maximum value from all
$G_i$:
\begin{align*}
 L = \max(\lbrace G_i\rbrace) \quad \forall i \in  \lbrace1,\cdots,N\rbrace
\end{align*}

In a final step $L$ is inverted, such that the lines are dark and the rest is
white. The result can be seen on the right side in \autoref{fig:sketch-steps}.

\begin{figure}[htb]
  \centering
  \includegraphics[width=0.4\textwidth]{images/sketch-steps.png}
  \caption{The intermediate results when calculating the line sketches. Left:
  input image, Center: Gradient image, Right: result $L$.}
  \label{fig:sketch-steps}
\end{figure}

\subsection{Shading}
The other important step in creating a believable pencil sketch image from a
natural image is to produce a hatching texture to create the shading. This is
done in two steps. First the histogram of the input image $I_g$ is matched to a
histogram model that was derived in \cite{mainPaper}. This way the tone
distribution is forced to correspond to tone distributions that were measured in
real pencil sketch images. Then the image of a given hatching pattern is used to
render the hatching texture for the input image. 

\paragraph{Histogram Matching}
Tones in natural images do not follow any specific pattern. In pencil drawings
however the tones are basically created only by two basic tones, namely the
white paper and the graphite strokes in different strengths. Heavy strokes are
used in very dark areas, mid tone strokes are used to produce impression of 3D
layered information and in bright areas the paper is just left white.
\autoref{fig:real-histograms} shows the tone distributions of some real pencil
sketches. One can easily see the three regions, the peak in the dark regions,
which represent the heavy strokes, the constant distribution in the mid tones,
which are used for the layering and very much bright pixels, originating from
the white paper, that was just left blank.

\begin{figure}[htb]
  \centering
  \includegraphics[width=0.4\textwidth]{images/real-histograms.png}
  \caption{Examples for real pencil sketches and their measured tone
    distributions. Note: This image was taken from \cite{mainPaper}}
  \label{fig:real-histograms}
\end{figure}

\cite{mainPaper} used this observation to create a parametric histogram model
for pencil drawings which consists of three functions, which represent those
three tone levels:

For the bright part of the histogram they use a Laplacian distribution with a
peak at the brightest value. This adds some variation in the bright areas, which
originate from slight illumination variances or the use of a eraser.

\begin{align}
  p_1(v) = \frac{255}{\sigma_b} e ^{-\frac{255-v}{\sigma_b}} \label{eq:p_1}
\end{align}

The parameter for this function is just $\sigma_b$, which controls the sharpness of
the function. This distribution can be seen on the very right of the histograms
in \autoref{fig:real-histograms}.

The mid layer is composed of strokes with different pressures and therefore in
different gray levels. So the distributions of those gray levels is equally
distributed as indicated in the histograms in \autoref{fig:real-histograms}. To
represent this part a constant function was chosen to use all those possible
gray levels.

\begin{align}
  p_2(v) = \begin{cases} \frac{1}{u_b - u_a} & \text{if } u_d < v \leq u_b\\
    0 & \text{otherwise}
  \end{cases} \label{eq:p_2}
\end{align}

The controlling parameters for this function are the range boundaries $u_d$
and $u_b$.

Finally the dark region which shows up as this bell shaped peak in the dark
regions in \autoref{fig:real-histograms} is represented as a Gauss-curve. The
position and shape of the dark regions depend on the maximum pressure an artist is using,
and the softness of the pencil that is used.

\begin{align}
  p_1(g) = \frac{1}{\sqrt{2\pi \sigma_d}} e^{-\frac{(v-\mu_d)^2}{2\sigma_d^2}} 
  \label{eq:p_3}
\end{align}

The width of the bell is controlled with the parameter $\sigma_d$ and the
position with $\mu_d$.

Plots for the three functions can be seen in \autoref{fig:p}.

\begin{figure}[htb]
  \centering
  \includegraphics[width=0.4\textwidth]{images/p_i.png}
  \caption{Plots of the three functions $p_i$. Note: picture taken from
    \cite{mainPaper}.}
  \label{fig:p}
\end{figure}

The final tone distribution is now simply composited out of those three
function by creating a weighted sum from $p_1$, $p_2$ and $p_3$:
\begin{align}
  p(v) = \frac{1}{Z} \sum_{i=1}^{3}\omega_i p_i(v)
  \label{eq:p}
\end{align}
Where $Z$ is a normalization factor to make $\int_0^{255}p(v)dv = 1$ and the
$\omega_i$ are weighting parameters which can used to weight the importance of
the functions.

In \cite{mainPaper} they learned the parameters for those functions from a set
of different styled pencil sketches using Maximum Likelihood Estimation. We skipped this part and left those
parameters to be controlled by the user.

Histogram Matching is used to apply the tone distribution from \autoref{eq:p} to
the input image. The result $J$ is then used to calculate the hatching texture.
$J$ can be seen on the right side of \autoref{fig:hist-result}.

\begin{figure}[htb]
  \centering
  \includegraphics[width=0.5\textwidth]{images/tone-result.png}
  \caption{Left input. Right: Result of Histogram matching for minions.}
  \label{fig:hist-result}
\end{figure}

\paragraph{Texturing}
The filter uses a men made pencil hatching pattern $H$. A human repetitively draws
at the same position to generate the correct tone in the hatching texture. A
exponential function can be used to simulate this process of placing multiple
strokes at the same position: $H(x)^{\beta(x)} \approx J(x)$. This corresponds
to drawing $H$ $\beta$ times at the same position to approximately match
the tone that is dictated by our tonal map $J$. In the logarithmic domain this
boils down to $\beta(x) \ln\left( H(x) \right) \approx \ln\left( J(x)
\right)$.

Just solving for $\beta$ in this equation is going to destroy the hatching
pattern because $\beta$ can be calculated for each pixel independently, such
that in the end $H^{\beta} = J$. Therefore a smoothness constraint is
introduced:
\begin{align}
  \beta = \arg \min_{\beta}\norm{\beta\ln(H) - \ln(J)}_2^2 + \lambda
  \norm{\nabla \beta}_2^2
  \label{eq:beta}
\end{align}
The smoothness weighting factor $\lambda$ can be used to determine how strong
the hatching pattern $H$ will show through in the result.

The hatching texture $T$ is then calculated as the pixelwise exponentiation
\begin{align*}
  T = H^{\beta}
\end{align*}
\autoref{fig:texture-result} shows the result for the minion.

\begin{figure}[htb]
  \centering
  \includegraphics[width=0.5\textwidth]{images/texture-result.png}
  \caption{Left: $J$. Right: Result of texture rendering $T$.}
  \label{fig:texture-result}
\end{figure}



\paragraph{Combining Results}
Finally the results from the line sketching $L$ and the hatching texture $T$ is
combined to the finished pencil drawing $R$ by simply multiplying the images
pixelwise:
\begin{align*}
  R = L  \cdot T
\end{align*}

If desired it is also possible to create colored pencil drawings using the $YUV$
decomposed image $I_{yuv}$ from the beginning and replace the $Y$ channel with
$R$. The resulting RGB-image can then easily be calculated by a color space
transformation.

In \autoref{fig:final-result} both, grayscale and colored results are shown.

\begin{figure}[htb]
  \centering
  \includegraphics[width=0.5\textwidth]{images/final-result.png}
  \caption{combination of the results from the line drawing and shading.}
  \label{fig:final-result}
\end{figure}


\section{GPU Implementation} \label{gpu-implementation}
Pipeline (Andreas)

\subsection{Scetch Filter}
Raphael
\subsection{Histogram}
Andreeas
\subsection{Histogram Matching}
Stefan
\subsection{Texturing}
Equation dings Raphael


\section{Performance} \label{performance}
TODO


\section{Future Work} \label{future-work}
TODO


\section{Conclusion}
We used modern technologies for parallel computing on GPUs to implement
an image filter to create pencil sketches from real images based on the
methods of a published paper.
In the course of this, common operations like color mode conversion,
gradient calculation, convolutions, histogram creation, and histogram
matching have been implemented as parallel working algorithms.
Our benchmarks results show that the speed of the image filter would even
be sufficient to apply it in real time for videos. This result is clearly an
indicator for a successful and efficient GPU implementation of the image
filter.



\bibliographystyle{alpha}
\bibliography{bibliography}


\end{document}

