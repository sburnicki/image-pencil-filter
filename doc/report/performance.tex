\section{Performance} \label{performance}
An important topic for the GPU implementation is of course the performance.
The described algorithm was implemented in a program that loads an image
from JPEG file, applies the filter to it, and saves it again to file.
Profiling this program showed very fast that loading the image, allocating the
needed buffers, copying it to GPU, downloading it afterwards, and saving it
as JPEG to disk take the most time by far.

As these are factors that aren't under our control, we chose to measure
the time it takes to apply the implemented filter, and ignore the mentioned
overhead in our measurements. This approach is legitimate, as in time critic
applications where the filter would be needed to be applied to multiple images,
the loading of the image and copying to GPU could be done while the GPU
is still busy with the previous image. Also the buffers can be reused,
such that their initialization doesn't really count.

We measured the time the GPU pipeline needs to execute by repeating the
filter 100 times on the same image to compensate for fluctuation. We chose to do
our benchmarks with image resolutions known from common video formats:
DVD/PAL (720x576), 720p (1280x720) and FullHD (1920x1080). The filter
application to the DVD quality test image took 30 ms, the 720p version
took 60ms and the FullHD image took 130ms which are all a lot faster
then the original implementation \cite{mainPaper}.

Although the implementation is fast enough to be applied in real-time when watching
a video in DVD quality, it doesn't make sense to use it as a video filter.
The reason for this is that static textures are used for the shading effect
which would be repeated in each video frame, such that the shading would look
very unnatural.

To determine why our implementation takes so "long" for a FullHD image,
we profiled its application to the test image to find bottlenecks. We found out
that one step in the "hatch texture calculation" part takes nearly half the
time for the whole filter application. The crucial part is to upload matrix
$A^*$ to the GPU. If this \texttt{cusp} matrix could be directly created on the
GPU from the data that is already there, the performance of the filter would
roughly be doubled.  Unfortunately, we didn't find a good solution to this problem
in the scope of this project.
