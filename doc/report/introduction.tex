\section{Introduction}
Throughout time artist have perfected the art of creating pencil drawings in all
thinkable styles, ranging from completely abstract to stunningly photo realistic.
However, such skills are rare, and only well trained artists or passionate hobby
artists are able to actually create pleasing looking images using nothing but
the eye, a paper, and a pencil. Today, in a time where cameras can be used to
capture a real world scene in a matter of milliseconds, with accurate proportions and
colors, without requiring special skills, the need for pencil sketches seem to be gone.
After all, photographs do not convey the same emotions that sketches do. There is a certain
timeless flair to pencil sketches that makes them just nice to look at.

For this reason designing an image filter which produces a pencil sketch from
a regular photograph is an exciting task. The filter described in
\cite{mainPaper} is such a filter, which consists of two stages: sketching the
outlines and drawing the shading. In the sketch stage the filter calculates several
line convolutions and the shading stage solves a huge linear system of
equations. Both tasks are computationally rather expensive and according to the
authors their implementation of the filter takes 2 seconds to run on a
$600\times 600$ image. \autoref{fig:teaser} shows input, intermediate
results and the final result of the filter.

This report presents our efforts to implement the entire filter on the GPU using
Cuda to gain real time results. The structure of the report is as follows: The
filter itself is described in \autoref{method}, then in
\autoref{gpu-implementation} we describe our implementation using Cuda.
\autoref{performance} shows our benchmark results for various input images
and in \autoref{future-work} some possible optimization and applications for our
implementation are pointed out.
