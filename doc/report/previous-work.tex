\section{Previous Work}
This section highlights the difficulty of solving the optimization problem in
\autoref{eq:error-function} by outlining other approaches and their
restrictions. The domain-transform paper
\cite{GastalOliveira2011DomainTransform} is left out on purpose, as it is
discussed in \autoref{domain-transform} in great detail.

The smoothness that is applied to a pixel by the regularization term depends on
the values of all the other pixels. Those dependencies often leads to non-convex
optimization problems, which are computationally infeasible for video volumes.

For this reason most existing methods which address the problem class in
\autoref{eq:error-function} concentrate on single-frame solutions, which are
computational tractable. However, single-frame solutions can not take account
for temporal consistency. Methods which also solve the temporal stability
problem typically use sliding windows to reduce computation costs
\cite{conf/psivt/HosniRBG11} and Kalman filtering to mix the results of
single-frame solutions \cite{conf/acivs/HoffkenOK11}. For those approaches it is
important to find the correct window size, as this parameter balances
computation costs and temporal smoothness. Poor parameter choices lead to
inconsistencies.

There also are methods which directly solve the global optimization using
precomputed optical flow as a representation of the frame-to-frame relations. An
example for this approach is \cite{Levin:2004:CUO:1015706.1015780}, which allows
to recolor a video based on sparse scribble input by the user. Those methods
generate high quality results but are computationally expensive and do not scale
well for higher resolution input images and videos.

