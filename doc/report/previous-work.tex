\section{Previous Work}
This work is based on \cite{mainPaper}, as this paper describes the filter that
is implemented. The filter requires quite some parameters. In the original paper
they use a parameter learning approach to automatically select those parameters.
In this work the filter is kept slightly simpler than in the origin paper, e.g.
the line filter is slightly simplified and the filter parameters are left as
user controllable options. 

Another GPU-sketch-filter is described in \cite{Lu:2010:IPS}, though it uses the
a regular shaderpipeline. Based on the neighborhood informations of each pixel
it calculates weather to place a stroke at this point or not. The strokes are
then rendered as stroke-textures. The strokes are made in three different detail
layers based on the gradientmagnitude. Furthermore it is possible to filter
animations using optical flow. The greatest difference to the filter that we
implement is that they calculate each individual pencil or brush stroke based on
the gradient and then render those strokes using stroke-textures. Our method
uses traditional filters to alter the original image, which produces better
pencil sketches with respect to the tonal appearance. However their filter is
more versatile, as it allows different painting stiles and techniques by
changing the stroke-textures and weighting the detail layers.


A partially GPU based method is described in \cite{Benard:2012:ASC}. However
they concentrate on rendering temporal coherent line drawing animations from a
given 3D-scene. First they find active contours that are then adapted throughout
the animation. The result pictures are rendered by parameterizing the and then
add a certain style to the path. While this method yields amazing results for
very stylistic animations, it needs 3D input data.
